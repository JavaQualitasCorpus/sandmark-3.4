\section{Adding an Obfuscator}
\SM\ is designed to make it easy to add a new obfuscation
algorithm. Assume that we want to add a new obfuscation
{\tt ReorderMethods}. The process would be the following:
\begin{enumerate}
   \item Create a new directory \url{sandmark/obfuscate/reorderMethods}.
   \item Create a new class  \url{sandmark/obfuscate/reorderMethods/ReorderMethods.java}.
   \item The \url{ReorderMethods} class should extend one of 
         the base classes \url{AppObfuscator} (if the algorithm works
         on the entire user program), \url{MethodObfuscator}
         (if the algorithm works one method at a time),
         or \url{ClassObfuscator} (if the algorithm works
         on one individual class at a time). Let's assume
         that our algorithm reorders methods within one
         class. \url{ReorderMethods} should therefore
         extend \url{ClassObfuscator}, which looks like
         this:
\begin{listing}{1}
package sandmark.obfuscate;
public abstract class ClassObfuscator extends GeneralObfuscator {
protected ClassObfuscator(String label) {
    super(label);
}
abstract public void apply(
    sandmark.util.ClassFileCollection cfc, String classname)
    throws Exception;
}
public String toString() {
    return "ClassObfuscator(" + getLabel() + ")";
}
}
\end{listing}
   \item The \url{ReorderMethods} class should look something
         like this:
\begin{listing}{1}
public class ReorderMethods extends sandmark.obfuscate.ClassObfuscator {
   public ReorderMethods(String label) {
      super(label);
   }
   public void apply(
       sandmark.util.ClassFileCollection cfc, String classname) throws Exception {
      // Your code goes here!
   }
}
\end{listing}
   \item Use {\tt BCEL} or {\tt BLOAT} to implement your obfuscation.
         The {\tt cfc} parameter represents the set of classes
         to be obfuscated. Use routines in \url{sandmark.util.ClassFileCollection}
         to open a class to be edited by {\tt BCEL} or {\tt BLOAT}.
   \item Type {\tt make} at the top-level sandmark directory (\url{smark}).
         The new obfuscation should be loaded automagically at runtime.
\end{enumerate}

\section{Obfuscation Configuration}
 To customize Obfuscation Algorithm Application
and Obfuscation Level Settings for a whole jar file ,or a class file and its methods  or a method alone
the configure button in the obfuscation panel is to be used .On clicking The Configure
button in the obfuscate panel , a new window pops up which is known
   as Obfuscation Dialog ,which helps the users to select specific obfuscation algorthims and
   obfuscation level .
\subsection {Method Summary}
\begin {listing}{2}
/*  Find if a particular algorithm is marked for obfuscation in the case
 *  of a class or jar file . Default is that all algorithms are to be applied
 *  for all classes and methods in the jar file .
 */
    public boolean IsMarked(String algo,String class_or_jarname);

/*  Find if a particular algorithm is marked for obfuscation in the case
 *  of a method . Please specify the  signature enclosed in brackets.
 */
    public boolean IsMarked(String algo,String classname,String methodname,
         String signature);

/*  Obtain the obfuscation level for a class or jar file. Possible String
 *  values are high , medium, low,none. Default is high .
 */
    public String getLevel(String class_or_jarname) ;

/*  Obtain the obfuscation  level for a method .The  signature should
 *   be enclosed in brackets .Possible values are high ,medium,low,none.
 *   Default is high .
 */
    public String getLevel(String classname,String methodname,String signature)
\end {listing}







