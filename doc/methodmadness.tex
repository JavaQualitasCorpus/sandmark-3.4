\algorithm{The Method 2R Madness Obfuscation Algorithm}%
          {Martin Stepp and Kelly Heffner}
          

\section{Introduction}
The Method 2R Madness Obfuscation (M2M) algorithm is based on the idea that
useful information about program organization lies in the parameter list
and separation of methods.  In order to destroy that information, the M2M 
obfuscator performs a six pass obfuscation:
\begin{enumerate}
\item In pass one, all fields and methods are made public.
\item In pass two, each dynamic method body are moved into a static helper
method.  The dynamic shell is simply a call on this static helper.
\item In pass three, all primitive types residing in a static method signature
are promoted to their respective object wrapper types.
\item In pass four, the parameter lists of all static methods are scrambled.
\item In pass five, the parameter lists of all static methods are condensed into
one large \verb/Object []/ array.
\item In pass six, static methods in the same class with the same signature are
merged into one method.
\end{enumerate}

\section{Pass Two - Method Splitting}
As mentioned above, the second phase of the M2M obfuscation splits each dynamic method
into two parts.  The first part is the same dynamic method, except that the body
of the method is changed to a simple call on a static helper method.  The first parameter
to this static method is the actual \verb/this/ reference which was implicitly argument
zero in the dynamic method.  Take for example the following class to computer Ackermann's 
function:
\begin{verbatim}
public class Ackermann 
{
    public static void main(String[] args) 
    {
      int num = Integer.parseInt(args[0]);
      System.out.println("Ack(3," + num + "): " + new Ackermann().Ack(3, num));
      }
    
    public int Ack(int m, int n) 
    {
      return (m == 0) ? (n + 1) : ((n == 0) ? Ack(m-1, 1) :
                  Ack(m-1, Ack(m, n - 1)));
    }
}
\end{verbatim}

After being split into a static section by the StaticSplit obfuscator phase, and then run through
the SourceAgain decompiler, the code looks as follows:

\begin{verbatim}
// 
// SourceAgain (TM) v1.10i (C) 2001 Ahpah Software Inc
// 

import java.io.PrintStream;

public class Ackermann {

    public static void main(String[] as)
    {
        int i = Integer.parseInt( as[0] );

        System.out.println( "Ack(3," + i + "): " + (new Ackermann().Ack( 3, i )) );
    }

    public static int Ackintint0(Ackermann this, int arg0, int arg1)
    {
        return (arg0 == 0) ? arg1 + 1 : (arg1 == 0) ? Ack( arg0 - 1, 1 ) : Ack( arg0 - 1, Ack( arg0, arg1 - 1 ) );
    }

    public int Ack(int arg0, int arg1)
    {
        return Ackintint0( this, arg0, arg1 );
    }
}
\end{verbatim}

\section{Pass Three - Primitive Promoting}
The third phase of the M2M obfuscation consists of promoting the argument list and return type of 
the static methods so that all primitive typed arguments or returns are now reference types.  A restriction
of reference only parameters and returns allows for easier parameter reordering and array parameterize,
and this primitive promoting obfuscation makes it possible for more static methods to fit the ``reference
only" criteria for reordering.  We will again use the Ackermann example from above, and tack on the 
primitive promoting pass.

\begin{verbatim}
// 
// SourceAgain (TM) v1.10i (C) 2001 Ahpah Software Inc
// 

import java.io.PrintStream;

public class Ackermann {

    public static void main(String[] arg0)
    {
        int i = Integer.parseInt( arg0[0] );

        System.out.println( "Ack(3," + i + "): " + (new Ackermann().Ack( 3, i )) );
    }

    public static Integer get0(Ackermann this, Integer arg0, Integer arg1)
    {
        return new Integer( (arg0.intValue() == 0) ? arg1.intValue() + 1 : (arg1.intValue() == 0) ? Ack( arg0.intValue() - 1, 1 ) : Ack( arg0.intValue() - 1, Ack( arg0.intValue(), arg1.intValue() - 1 ) ) );
    }

    public int Ack(int arg0, int arg1)
    {
        return get0( (Ackermann) new Object[3][0], (Integer) new Object[3][1], (Integer) new Object[3][2] ).intValue();
    }
}

\end{verbatim}
Notice that the decompiler has a problem interpreting that the arguments to method calls were promoted using a
temporary array of Objects.

\section{Pass Four - Parameter Reordering}
The parameter reorderer adds an extra dose of confusion to the algorithm, so that there is no direct mapping 
between the order of the nonstatic stub methods, and their static helpers.  This also shuffles the \verb/this/
reference out of slot zero, making our static helpers less obvious.

\begin{verbatim}
// 
// SourceAgain (TM) v1.10i (C) 2001 Ahpah Software Inc
// 

import java.io.PrintStream;

public class Ackermann {

    public static void main(String[] arg0)
    {
        int i = Integer.parseInt( arg0[0] );

        System.out.println( "Ack(3," + i + "): " + (new Ackermann().Ack( 3, i )) );
    }

    public static Integer get0(Integer this, Ackermann arg0, Integer arg1)
    {
        return new Integer( (arg1.intValue() == 0) ? intValue() + 1 : (intValue() == 0) ? arg0.Ack( arg1.intValue() - 1, 1 ) : arg0.Ack( arg1.intValue() - 1, arg0.Ack( arg1.intValue(), intValue() - 1 ) ) );
    }

    public int Ack(int arg0, int arg1)
    {
        return get0( (Integer) new Object[3][0], (Ackermann) new Object[3][1], (Integer) new Object[3][2] ).intValue();
    }
}
\end{verbatim}

\section{Pass Five - Signature Bludgeoning}
The bludgeon takes the final step in confusing the signatures of methods, by making all of the static
method signatures the \emph{same}, ``(L[java/lang/Object;)Ljava/lang/Object;".  This prepares the classes
for the final pass, which is to combine any methods with the same signature (which is now all static methods).

\begin{verbatim}
// 
// SourceAgain (TM) v1.10i (C) 2001 Ahpah Software Inc
// 

import java.io.PrintStream;

public class Ackermann {

    public static void main(String[] arg0)
    {
        int i = Integer.parseInt( arg0[0] );

        System.out.println( "Ack(3," + i + "): " + (new Ackermann().Ack( 3, i )) );
    }

    public static Object get0(Object[] this)
    {
        return new Integer( (((Integer) this[2]).intValue() == 0) ? ((Integer) this[0]).intValue() + 1 : (((Integer) this[0]).intValue() == 0) ? ((Ackermann) this[1]).Ack( ((Integer) this[2]).intValue() - 1, 1 ) : ((Ackermann) this[1]).Ack( ((Integer) this[2]).intValue() - 1, ((Ackermann) this[1]).Ack( ((Integer) this[2]).intValue(), ((Integer) this[0]).intValue() - 1 ) ) );
    }

    public int Ack(int arg0, int arg1)
    {
        return ((Integer) get0( new Object[] { (Integer) new Object[3][0], (Ackermann) new Object[3][1], (Integer) new Object[3][2] } )).intValue();
    }
}
\end{verbatim}

\section{Pass Six - Method Merging}
The method merger combines static methods with the same signature into one method, that is multiplexed on an extra
\verb/int/ parameter.  Consider the code for the original Ackermann, with an extra static method added in:
\begin{verbatim}
public class Ackermann 
{
    public static void main(String[] args) 
    {
      int num = Integer.parseInt(args[0]);
      System.out.println("Ack(3," + num + "): " + new Ackermann().Ack(3, num));
      }
    
		public static void hi(long p)
		{
			System.out.println(p);
		}

    public int Ack(int m, int n) 
    {
      return (m == 0) ? (n + 1) : ((n == 0) ? Ack(m-1, 1) :
                  Ack(m-1, Ack(m, n - 1)));
    }
}
\end{verbatim}
After a pass through all five other phases, the two static methods (which now share an Object[] to Object signature)
are merged together as follows:

\begin{verbatim}
// 
// SourceAgain (TM) v1.10i (C) 2001 Ahpah Software Inc
// 

import java.io.PrintStream;

public class Ackermann {

    public static void main(String[] arg0)
    {
        int i = Integer.parseInt( arg0[0] );

        System.out.println( "Ack(3," + i + "): " + (new Ackermann().Ack( 3, i )) );
    }

    public int Ack(int arg0, int arg1)
    {
        return ((Integer) keldaANDmartio0( new Object[] { (Integer) new Object[3][0], (Integer) new Object[3][1], (Ackermann) new Object[3][2] }, 1 )).intValue();
    }

    public static Object keldaANDmartio0(Object[] keldaANDmartio0, int switchVar)
    {
        if( 1 == switchVar )
            return new Integer( (((Integer) keldaANDmartio0[1]).intValue() == 0) ? ((Integer) keldaANDmartio0[0]).intValue() + 1 : (((Integer) keldaANDmartio0[0]).intValue() == 0) ? ((Ackermann) keldaANDmartio0[2]).Ack( ((Integer) keldaANDmartio0[1]).intValue() - 1, 1 ) : ((Ackermann) keldaANDmartio0[2]).Ack( ((Integer) keldaANDmartio0[1]).intValue() - 1, ((Ackermann) keldaANDmartio0[2]).Ack( ((Integer) keldaANDmartio0[1]).intValue(), ((Integer) keldaANDmartio0[0]).intValue() - 1 ) ) );
        if( 0 != switchVar )
            ;
        System.out.println( ((Long) keldaANDmartio0[0]).longValue() );
        return null;
    }
}
\end{verbatim}