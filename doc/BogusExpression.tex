\algorithm{Project VIII: Class-Splitting Obfuscations}%
	  {Team : Ashok Purushotham and RathnaPrabhu Rajendran}

\section{Mission Document}
	For our final project we would like to implement a class-splitting
obfuscator for Java . 

\section{Basic Idea}
The basic idea behind the project is that
there are two important object  splitting techniques. The first
splits each object into two, such that some fields belong to the
first subpart, the remainder to the second subpart.


\section{How we plan to implement}
        In the bytecode every "new"-operation turns into two "new"s
and every pointer operation (getfield and setfield) turns into two
pointer operations. The second splitting technique splits at the
class level. A class C is broken into two classes C 1 and C 2 ,
such that C 2 inherits from C 1 . C 1 has fields and methods that
only refer to themselves, whereas C 2 has fields and methods that
can refer to themselves as well as fields and methods in C1 .
Bytecode references to C will have to be replaced with references
to C 2 .

        The reason we want to implement this algorithm is that, to
the best of our knowledge, it hasn't been implemented before. We
would like to prove this by implementing and evaluating it. The
implementation will be done within the SandMark framework. This will
allow us to take advantage of the many excellent, well-documented
libraries that are available within this tool.

\section {Project Tasks}
        We believe this will be a difficult project with many potential
pitfalls. First of all, the paper talks shallowly about our project and
nothing else ,so we need to work out a reasonable algorithm. Second we
can't simply split classes ,but splitting should be based on inheritance
principles and understanding all of them and implementing them as a single
task will be an uphill task .We therefore identify the following subtasks:
1. Understanding of concepts of splitting classes,going thourgh all
   relevant material in Sandmark and Collberg's paper.
2. Design
3. Discussing our design and implementation algorithms with Prof.Collberg
   and Coding
4. Testing and evaluation.
5. Documentation.

\section{Reference}
        The reference for this project will be Collberg: Breaking
Abstractions and Unstructuring Data Structures.

