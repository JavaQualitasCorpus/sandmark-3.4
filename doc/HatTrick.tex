\algorithm{The HatTrick Watermarking Algorithm}%
          {Danny Mandel and Anna Segurson}

\section{Introduction}

This algorithm embeds a static watermark in a java *.jar file through
the sandmark interface. It first converts the watermark (which is a String)
to a number. This number is then encoded into the java byte code by adding
bogus local variables to one method of one class in the *.jar file. The
class in which the method is in is marked by a extra field variable that
contains the name of the method.

Each number of the watermark is mapped to a specific type:

\begin{center}
\begin{tabular}{| c | c |}
  \hline
  Number & Type \\
  \hline
  0 & \texttt{java.util.GregorianCalendar} \\
  \hline
  1 & \texttt{java.lang.Thread} \\
  \hline
  2 & \texttt{java.util.Vector} \\
  \hline
  3 & \texttt{java.util.Stack} \\
  \hline
  4 & \texttt{java.util.Date} \\
  \hline
  5 & \texttt{java.io.InputStream} \\
  \hline
  6 & \texttt{java.io.ObjectInputStream} \\
  \hline
  7 & \texttt{java.lang.Math} \\
  \hline
  8 & \texttt{java.io.OutputStream} \\
  \hline
  9 & \texttt{java.lang.String} \\
  \hline
\end{tabular}
\end{center}

\hfill \\
A local variable of the proper type is created for each digit of the watermark
and the digit's location in the watermark is stored in the name of the bogus
local variable. The rightmost digit is position $0$.


The code of this algorithm
resides in \url{sandmark.watermark.hattrick}.


\section{Embedding}
The watermark is embedded in the first non-abstract method found in 
the first class file
of the jar file. The name of the method is recorded in the name of a field with
the value of a hockey player's name. These are the local variables added to
a class when the watermark ``Howdy!" was embedded:
\begin{listing}{1}
   Math yzerman$0 = new Math();
   String yzerman$1 = new String();
   ObjectInputStream yzerman$2 = new ObjectInputStream();
   Stack yzerman$3 = new Stack();
   InputStream yzerman$4 = new InputStream();
   Stack yzerman$5 = new Stack();
   OutputStream yzerman$6 = new OutputStream();
   InputStream yzerman$7 = new InputStream();
   InputStream yzerman$8 = new InputStream();
   OutputStream yzerman$9 = new OutputStream();
   Thread yzerman$10 = new Thread();
   Thread yzerman$11 = new Thread();
   Thread yzerman$12 = new Thread();
   ObjectInputStream yzerman$13 = new ObjectInputStream();
   Stack yzerman$14 = new Stack();
\end{listing}
This is the field that is added:
\texttt{public static final String hat<init>Trick = ``Vrbata'';}
where \texttt{<init>} is the name of the method the watermark is in.

\section{Recognition}
The watermark is recognized when the proper field is present which stores the
name of the method. Every local variable that contains the \texttt{SECRET\_NAME}
in that method is part of the watermark.

In the example listed above \texttt{yzerman} is the secret name. To recover the
number stored by the \texttt{SECRET\_NAME} variables, the algorithm initializes
a number, \texttt{wm}, to zero then loops 
through all of the locals in the method and adds the mapped value of the
variable times 10 to the power of the number after the \$ to \texttt{wm}.
After the number is recovered it is converted back to the watermark string.

